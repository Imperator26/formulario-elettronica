\documentclass{article}

\usepackage[a4paper, total={15cm, 25cm}]{geometry}

\usepackage[utf8]{inputenc}
\usepackage{amsmath}
\usepackage{amssymb}

\title{Formulario di Elettronica}
\author{}
\date{}

\begin{document}
\maketitle

\section*{Silicio}
Legge di Azione di Massa:
\begin{equation*}
n \cdot p = h(T)
\end{equation*}
per il silicio intrinseco ad una temperatura di $300K$ $n_i = 1.4 \times 10^{10} \frac{\text{elettroni liberi}}{cm^3}$.\\
No. di elettroni con livello energetico $E$:
\begin{equation*}
K(E)=P(E) \cdot g(E)
\end{equation*}
\begin{equation} \label{eq:no_n}
n = gkTe^\frac{(E_C-E_F)}{kT}
\end{equation}
\begin{equation} \label{eq:no_p}
p = gkT e^\frac{(E_F-E_V)}{kT}
\end{equation}
Per il silicio intrinseco $n=p$ quindi:
\begin{equation*}
E_{Fi} = \frac{E_V + E_C}{2}
\end{equation*}
Allora la \eqref{eq:no_n} e la \eqref{eq:no_p} si possono riscrivere in funzione dell'$E_{Fi}$:
\begin{equation*}
n = n_i e^\frac{(E_F-E_{Fi})}{kT}
\end{equation*}
\begin{equation*}
p = n_i e^\frac{(E_{Fi}-E_F}{kT}
\end{equation*}
Definiamo energia potenziale nella rispettiva zona (n, p) e di built-in:
\begin{equation*}
\varphi_n = E_F-E_{Fi} = kT \log \left( \frac{N_D}{n_i} \right)
\end{equation*}
\begin{equation*}
\varphi_p = E_{Fi}-E_F = kT \log \left(\frac{N_A}{n_i}\right)
\end{equation*}
\begin{equation*}
\varphi_{bi} = \varphi_n + \varphi_p = kT \log \left(\frac{N_DN_A}{n_i^2}\right)
\end{equation*}
Definiamo Potenziale di Built-In:
\begin{equation*}
V_{bi} = \frac{\varphi_{bi}}{q} = \frac{\varphi_n + \varphi_p}{q} = \frac{kT}{q} \log \left(\frac{N_DN_A}{n_i^2}\right)
\end{equation*}
Dalla sequente:
\begin{equation*}
\frac{dE}{dx} = \frac{\rho}{\varepsilon_{Si}}
\end{equation*}
si ricava il campo elettrico massimo:
\begin{equation} \label{eq:e_max}
E_{max} = \frac{qN_Ax_p}{\varepsilon_{Si}} = \frac{qN_Dx_n}{\varepsilon_{Si}}
\end{equation}
da cui si ottiene:
\begin{equation*}
N_Dx_n = N_Ax_p
\end{equation*}
Inoltre sappiamo che le differenze di potenziale ($\Delta V_n$ e $\Delta V_p$) delle due zone è il potenziale di built-in. Con ciò si possono ricavare le relazioni in funzione della $V_{bi}$ delle dimensioni delle zone di carica spaziale:
\begin{equation} \label{eq:zcs_n}
x_n = \sqrt{\frac{2\epsilon_{Si}}{q} \frac{N_A}{N_D} \frac{V_{bi}}{N_D+N_A}}
\end{equation}
\begin{equation} \label{eq:zcs_p}
x_p = \sqrt{\frac{2\varepsilon_{Si}}{q} \frac{N_D}{N_A} \frac{V_{bi}}{N_D+N_A}}
\end{equation}
Dimensione totale della z.c.s.:
\begin{equation*}
x_d = x_n + x_p = \sqrt{\frac{2\varepsilon_{Si}}{q} \frac{N_D+N_A}{N_DN_A} V_{bi}}
\end{equation*}
Nel caso di una giunzione unilatera si ha che la z.c.s. è prevalentemente estesa nella zona meno drogata.

\section*{Polarizzazione Inversa}
Viene applicata una tensione alla giunzione $pn$ affinché le z.c.s. si allarghino. Dalla \eqref{eq:zcs_n} e dalla \eqref{eq:zcs_p} le dimensioni delle cariche spaziali diventano quindi:
\begin{equation*}
x_n = \sqrt{\frac{2\varepsilon_{Si}}{q} \frac{N_A}{N_D} \frac{1}{N_D+N_A} (V_{bi} + V_{ext})}
\end{equation*}
\begin{equation*}
x_p = \sqrt{\frac{2\varepsilon_{Si}}{q} \frac{N_D}{N_A} \frac{1}{N_D+N_A} (V_{bi} + V_{ext})}
\end{equation*}

\subsection*{Capacità di giunzione}
Se le regioni si allargano, vuol dire che delle cariche libere si sono spostate generando una corrente.
Il comportamento di una giunzione può essere approssimato a quella di un condensatore a facce piane parallele:
\begin{equation*}
C_j = \epsilon_{Si} \cdot \frac{A}{x_d}
\end{equation*}

\subsection*{Limiti massimi di $V_{ext}$ in inversa}
\subsubsection*{Effetto tunnel}
\subsubsection*{Effetto valanga}
Si definisce tensione di breakdown $V_{BD}$ la tensione massima che la giunzione può sostenere prima che uno dei due effetti sopracitati prenda il sopravvento.
Per una giunzione unilatera ($N_D>>N_A$) attraverso la \eqref{eq:e_max} si ricava la relazione della tensione di breakdown:
\begin{equation*}
V_{BD} = \frac{E_{BD}^2 \varepsilon_{Si}}{2qN_A} - V_{bi}
\end{equation*}

\section*{Polarizzazione Diretta}
In questo caso la tensione $V_{ext}$ applicata abbassa la barriera di potenziale tra le due zone drogate tanto da creare un passaggio di corrente significativo che nel caso di polarizzazione inversa è trascurabile.
Definiamo densità di corrente:
\begin{equation*}
J_n = qD_n \frac{dn}{dx}
\end{equation*}
Con $D_n$ coeff. di diffusione.
\\
Relazione di Einstein che lega il coeff. di diffusione e la mobilità:
\begin{equation*}
D_n = \frac{kT}{q} \mu_n
\end{equation*}

\section*{Fotodiodo}
Ovvero un diodo polarizzato in inversa illuminato da un fascio di luce.
La radiazione che colpisce il diodo può liberare degli elettroni che, scorrendo nel circuito, andranno a ricombinarsi producendo una corrente pari a:
\begin{equation*}
I_\nu = [\frac{\#fotoni}{area \cdot t} h\nu]
\end{equation*}
L'assorbimento della radiazione è un fenomeno di volume per cui segue la relazione:
\begin{equation*}
I_\nu(x) = I_\nu(0) e^{-\alpha x}
\end{equation*}
con $\alpha = \frac{1}{L_A}$ cioè il reciproco della lunghezza di assorbimento.
\\
La resistenza dovuta dalle zone neutre (dove non c'è rigenerazione) vale:
\begin{equation*}
R_{(n, N_D)}^{neutra} = \frac{1}{N_D q \mu_n} \frac{w_n}{S}
\end{equation*}

\section*{Fotodiodo pin}
Il fotodiodo pin consiste in una giunzione con del silicio intrinseco tra le regioni p e n. Viene utilizzato nei diodi che devono sostenere alte tensioni di lavoro perché praticamente tutta la $V$ cade su $w_i$.

\section*{Diodo}
Per calcolare la corrente passante si distingue tra due casi:
Definiamo $\tau_{tr} = \frac{x_p^2}{2D_n}$ il tempo di transito di un $e^{-}$ e $\tau_n$ il tempo di vita dell'elettrone cioè il tempo medio di ricombinazione.

\subsubsection*{Base corta}
- L'approx. di base corta trascura la ricombinazione cioè quando $\tau_{tr}<<\tau_n$. Da cui $w_p<<L_n$ con $L_n = \sqrt{D_n \tau_n}$ lunghezza di diffusione (distanza percorsa prima di ricombinarsi).
La corrente totale è quindi:
\begin{equation*}
I_{tot} = I_n + I_p = Aqn_i^2 \left[ \frac{D_n}{N_A w_p^\prime} + \frac{D_p}{N_D w_n^\prime} \right] \left[ e^\frac{qV_{ext}}{kT} - 1 \right] = I_S \cdot \left( e^\frac{qV_{ext}}{kT} - 1 \right)
\end{equation*}
con $w_n^\prime = w_n - x_n$ e $w_p^\prime = w_p - x_p$ cioè le zone neutre.\\
Dalla relazione precedente sembrerebbe che la corrente diminuisca all'aumentare della temperatura, in realtà c'è da considerare che 
\begin{equation*}
n_i^2 = np \propto e^{-\frac{(E_C - E_F)}{kT}} - e^{-\frac{(E_F - E_V)}{kT}} = e^{-\frac{E_{gap}}{kT}}
\end{equation*}
quindi c'è una relazione di tipo esponenziale tra l'aumento dei portatori liberi di carica e l'aumento della temperatura che fa aumentare la corrente.

\subsubsection*{Base lunga}
- Il caso del diodo a base lunga invece tiene conto della ricombinazione degli elettroni. La relazione della corrente è analoga a quella per il diodo a base corta ma $w_p$ e $w_n$ sono rispettivamente approssimabili con $L_p$ e $L_n$. 
\begin{equation*}
I_{tot} = Aqn_i^2 \left[ \frac{D_n}{N_A L_p^\prime} + \frac{D_p}{N_D L_n^\prime} \right] \left[ e^\frac{qV_{ext}}{kT} - 1 \right] = I_S \cdot \left( e^\frac{qV_{ext}}{kT} - 1 \right)
\end{equation*}

\section*{Transistore Bipolare (BJT)}
\textbf{Interdizione}\\
Entrambe le giunzioni sono polarizzate inversamente:\\
\hspace*{2cm} $V_{BE}, V_{BC} < 0.7$\\
\hspace*{2cm} $I_B \approx 0$\\
\\
Regione attiva inversa\\
Quando $V_{BE} < 0.7V$ e $V_{BC} > 0.7V$.
Regione attiva diretta\\
Appena la tensione applicata raggiunge la tensione di soglia la giunzione BE è accesa mentre quella BC è spenta. Per valori della tensione applicata maggiori di 0.7V vale la seguente:
\begin{equation*}
I_B = \frac{V - (0.7)}{R_B}
\end{equation*}
Quindi trascurando l'effetto Early vale:
\begin{equation*}
I_C = \beta_F I_B
\end{equation*}
\\
\textbf{Saturazione}\\
\hspace*{2cm} $V_{BE}, V_{BC} \geqslant 0.7$\\

\section*{MOSFET}
\subsection*{MOS - Metal Oxide Semiconductor}
Tensione di flat-band:
\begin{equation*}
V_{fb} = -\Delta V = - \left( \frac{E_{gap}}{2q} + \varphi_p \right)
\end{equation*}
Ovvero la tensione che applicata tra gate e substrato fa coincidere le $E_F$ annullando il campo elettrico in direzione verticale. Ad esempio se il substrato è di tipo $p$ e nessuna tensione è applicata tra gate e substrato, si ha una differenza di potenziale tra metallo ($-$) e substrato ($+$). Aumentando la tensione esterna tra i due strati si raggiunge una condizione di equilibrio dove la tensione totale è nulla. proseguendo oltre e superando la tensione di soglia si ottiene un'inversione del substrato, che diventa equivalente ad una zona $n$, e si crea un canale conduttivo al di sotto dell'ossido composto da cariche libere. La tensione di flat-band è negativa se il substrato è di tipo $p$, viceversa se di tipo $n$.\\
\\
Tutto quanto detto a seguire vale per la tensione di gate-source e la tensione del bulk a massa. Nel paragrafo Effetto body toglieremo questo vincolo.\\
\\
Si definisce tensione di soglia $V_T$ la tensione per cui si ha la condizione di inversione della zona drogata, ad esempio da zona $p$ diventa zona $n$ ($\varphi_p = \varphi_n$). Questa inversione fa si che si crei un canale conduttivo subito sotto l'ossido dove può scorrere una corrente.\\
Dalle seguenti relazioni
\begin{equation*}
V_{Si} = 2\varphi_p
\end{equation*}
\begin{equation*}
V_{ox} = \frac{1}{C_{ox}} \sqrt{2\varepsilon_{Si} q N_A 2\varphi_p}
\end{equation*}
\begin{equation*}
V_{ox} + V_{Si} = \frac{E_{gap}}{2q} + \varphi_p + V_T
\end{equation*}
si ricava la relazione per la tensione di soglia:
\begin{equation*}
V_T = -\left( \frac{E_{gap}}{2q} + \varphi_p \right) + 2\varphi_p + \frac{1}{C_{ox}}\sqrt{2\varepsilon_{Si} q N_A 2\varphi_p}
\end{equation*}
Se $V_{GS} < V_T$ ci si trova nella condizione di interdizione per cui non può scorrere corrente ($I_D \approx 0$). Infatti entrambe le giunzioni sono polarizzate in inversa.\\
\\
Se $V_{GS} > V_T$ allora è presente un canale conduttivo sotto l'ossido dove gli elettroni possono scorrere. La carica che si forma si può ricavare dall'approssimazione del MOS ad un condensatore:
\begin{equation*}
Q = -C_{ox} (V_{GS} - V_T)
\end{equation*}
con $C_{ox} = \frac{\varepsilon_{ox}}{t_{ox}}$ capacità dell'ossido per unità di area. $t_{ox}$ è lo spessore dell'ossido.

\begin{equation}	\label{eq:corrente_ch}
I_D = C_{ox} \mu_n \frac{W}{L} \left[ (V_{GS} - V_T)V_{DS} - \frac{1}{2}V_{DS}^2 \right]
\end{equation}
Per $V_{DS}$ piccole si può trascurare il termine quadratico ottenendo:
\begin{equation*}
I_D = \frac{V_{DS}}{R_{Ch}}
\end{equation*}
con $R_{Ch} = \frac{1}{\frac{W}{L} C_{ox} \mu_n (V_{GS} - V_T)}$  definita resistenza di canale. In questa condizione ($V_{DS} << V_{GS} - V_T$) si è in regime ohmico o lineare.\\
Per $V_{DS} >> V_{GS} - V_T$ entra in gioco anche il termine quadratico che non è più trascurabile quindi la corrente passante tende a diminuire parabolicamente. Calcoliamo il massimo:
\begin{equation*}
\frac{\partial I_D}{ \partial V_{DS}} = C_{ox} \mu_n \frac{W}{L} (V_{GS} - V_T - V_{DS}) = 0
\end{equation*}
si ha la corrente massima quando $V_{GS} - V_T = V_{DS}$, da cui:
\begin{equation}	\label{eq:corrente_sat}
I_D = \frac{1}{2} C_{ox} \mu_n \frac{W}{L} (V_{GS} - V_T)^2
\end{equation}


\subsection*{Modulazione per effetto canale}
Nel caso in cui ci si trovi nella situazione in cui $V_{DS} > V_{GS} - V_T + \Delta V_{DS}$ cioè con una tensione drain-source poco superiore a $V_{GS} - V_T$ di una quantità $\Delta V_{DS}$ piccola, il canale non attraverserà completamente il substrato ma ci sarà un punto $P$, dove termina la zona di cariche mobili.\\
\\
Quindi definiamo la lunghezza efficace $L_{eff} = L - \Delta L$ che è la lunghezza del canale. $\Delta L$ è la distanza necessaria per avere il termine di tensione $\Delta V_{DS}$ tenendo conto del campo elettrico presente $E_{max}$, quindi $\Delta L = \frac{\Delta V_{DS}}{E_{max}}$.\\
\\
Utilizzando la \eqref{eq:corrente_sat} sostituendo $L_{eff}$ si ottiene:
\begin{equation*}
I_D = \frac{1}{2} C_{ox} \mu_n \frac{W}{L_{eff}} (V_{GS} - V_T)^2
\end{equation*}
\begin{equation*}
I_D = \frac{1}{2} C_{ox} \mu_n \frac{W}{L} \left(1+\frac{\Delta L}{L} \right) (V_{GS} - V_T)^2 = \frac{1}{2} C_{ox} \mu_n \frac{W}{L} \left(1+\frac{\Delta V_{DS}}{E_{max} L} \right) (V_{GS} - V_T)^2
\end{equation*}
Dalla quale si evince che la corrente non è completamente indipendente dalla tensione $V_{DS}$.

\subsection*{Effetto body}
Si rimuove il vincolo che il source e il body siano allo stesso potenziale.
\begin{equation*}
V_T^* = V_{T_0} + \gamma \left( \sqrt{2\varphi_p + V_{SB}} - \sqrt{2\varphi_p} \right)
\end{equation*}
con $V_{T_0}$ la tensione di soglia quando $V_{SB} = 0$ e con $\gamma = \frac{\sqrt{qN_A 2\varepsilon_{Si}}}{C_{ox}}$ detto coefficiente di body.
\begin{equation*}
I_D = \frac{1}{2} C_{ox} \mu_n \frac{W}{L} \left[ V_{GS} - V_{T_0} + \gamma \left( \sqrt{2\varphi_p + V_{SB}} - \sqrt{2\varphi_p} \right) \right]^2
\end{equation*}

\subsection*{pMOSFET}
\textbf{Interdizione:}\\
\hspace*{2cm} $V_{SG} < |V_T|$\\
\hspace*{2cm} $I_D \approx 0$\\
\textbf{Acceso:}\\
\hspace*{2cm} $V_{SG} > |V_T|$\\
\hspace*{1cm} Regime ohmico:\\
\hspace*{2cm} $V_{DG} > |V_T|$\\
\hspace*{2cm} $I_D = C_{ox} \mu_n \frac{W}{L} \left[ (V_{SG} - |V_T|)V_{SD} - \frac{1}{2}V_{SD}^2 \right]$\\
\hspace*{1cm} Saturazione:\\
\hspace*{2cm} $V_{DG} < |V_T|$\\
\hspace*{2cm} $I_D = \frac{1}{2} C_{ox} \mu_n \frac{W}{L} (V_{SG} - |V_T|)^2$

\newpage
\section*{Elettronica Digitale}

\subsection*{Invertitore logico ($NOT$)}

\subsection*{Invertitore CMOS (complementary MOS)}
%Immagine inverter CMOS
%Descrizione del circuito
Analizziamo l'inverter facendo 3 casi:\\
Per $V_x = 0V$ si ha il pMOS acceso, essendo $V_{SG} = V_{DD} > |V_{Tp}|$, ma l'nMOS è spento. Essendo l'nMOS in regime di interdizione, la corrente che gli scorre attraverso è nulla. Per la legge di Kirchhoff la corrente che scorre attraverso il pMOS e l'nMOS deve essere la stessa. Di conseguenza, $I_D = 0A$.\\
Per $V_x = V_{DD}$ si ha la situazione inversa. Il pMOS è spento mentre l'nMOS è acceso. Ancora una volta la corrente che scorre nell'inverter è nulla.\\
Vediamo ora il caso in cui la tensione $V_x = V_{Tn} + V_\varepsilon$ ovvero un valore leggermente maggiore della tensione di soglia dell'nMOS. In questo caso si ha acceso sia l'nMOS sia il pMOS. %L'nMOS è in regime ohmico, il pMOS in regime saturo, quindi la corrente passante è dettata dall'nMOS (corrente più piccola tra le due). Aumentando la corrente $I_Dn$ si raggiunge la condizione in cui entrambe le correnti sono di saturazione.
Il circuito impone che $I_{Dp} = I_{Dn}$ da cui si può ricavare la tensione di threshold $V_{TH}$ alla quale avviene la commutazione tra segnale logico 1 e 0. Questa situazione avviene quando entrambi i MOS sono in regime di saturazione per cui:
\begin{equation*}
I_{Dn} = K_n (V_{GSn} - V_{Tn})^2 \stackrel{topologia}{=} K_p (V_{SGp} - |V_{Tp}|)^2 = I_{Dp}
\end{equation*}
sostituendo $V_{GSn} = V_{TH}$ e $V_{SGp} = V_{DD} - V_{TH}$ si ottiene
\begin{equation*}
\sqrt{K_n} (V_{TH} - V_{Tn}) = \sqrt{K_p} (V_{DD} - V_{TH} - |V_{Tp}|)
\end{equation*}
generalmente si lavora nella condizione $K_n = K_p$ e $V_{Tn} = V_{Tp}$. Sostituendo, ricavando $V_{TH}$ e semplificando si ottiene
\begin{equation*}
V_{TH} = \frac{V_{DD}}{2}
\end{equation*}

\subsubsection*{Analisi dinamica}
\subsubsection*{Tempi di commutazione}
\subsubsection*{Potenza dissipata}
\begin{equation*}
\overline{P_{dissipata}} = \overline{P_{\text{erogata da $V_{DD}$}}} = V_{DD} \cdot \overline{I_{DD}}
\end{equation*}
Siccome la corrente non scorre quando la tensione applicata è $V_{DD}$ o $0V$, è la commutazione che fa consumare energia. Trascuriamo l'andamento reale della corrente per semplificare i conti; quindi ragioniamo in termini di carica.
\begin{equation*}
\overline{P_{diss}} = V_{DD} \cdot \overline{I_{DD}} = V_{DD} \frac{\Delta Q}{T} = V{DD} \frac{\Delta V_y}{T} = V_{DD} \frac{C_L (V_{y,high} - V_{y,low})}{T} = V_{DD} C_L (V_{y,high} - V_{y,low})f
\end{equation*}
essendo $V_{y,high} = V_{DD}$ e $V_{y,low} = 0V$ si ha
\begin{equation*}
\overline{P_{diss}} = V_{DD}^2 C_L f
\end{equation*}
Quindi per minimizzare il consumo si cerca di diminuire $C_L$ attraverso la miniaturizzazione dei componenti. Inoltre si tende a lavorare a $V_{DD}$ piccole.

\subsubsection*{Potenza di cross-conduzione}
Supponiamo di collegare al morsetto $V_y$ del invertitore analizzato precedentemente un secondo invertitore CMOS.
- Porta NOR CMOS\\
-- Tempi di commutazione\\
- Memoria DRAM\\
-- Scrittura "0"\\
-- Scrittura "1"\\
-- Lettura\\
- Porta XOR (OR esclusivo)\\
-- Tempi di commutazione\\
-- Potenza dissipata\\

\newpage
\section*{Elettronica Analogica}
\textbf{Amplificatore lineare}\\
\hspace*{1cm} Amplificatore di tensione:\\
\hspace*{2cm} $V_{out} = A \cdot V_{in}$\\
\hspace*{1cm} Amplificatore di corrente\\
\hspace*{2cm} $I_{out} = A \cdot I_{in}$\\
\hspace*{1cm} Amplificatore di trans-resistenza\\
\hspace*{2cm} $V_{out} = A \cdot I_{in}$\\
\hspace*{1cm} Amplificatore di trans-conduttanza\\
\hspace*{2cm} $I_{out} = A \cdot V_{in}$\\
-- Struttura\\
- Amplificatore lineare di piccolo segnale\\
-- Analisi della polarizzazione\\
-- Analisi di piccolo segnale\\
- Source follower\\
-- Analisi della polarizzazione\\
-- Analisi di piccolo segnale\\
- Amplificatore con resistenza di degenerazione\\
-- Analisi della polarizzazione\\
-- Analisi di piccolo segnale\\
- Analisi nel dominio trasformato\\
-- Trasformata di Fourier\\
-- Trasformata di Laplace\\
-- Diagrammi di Bode\\
- Sistema retroazionato\\
-- Amplificatore invertente (retroazione negativa)\\
-- Amplificatore sommatore\\
-- Integratore\\
- Amplificatore Operazionale - imperfezioni in continua\\
-- Tensioni di offset\\
-- Correnti di bias\\
-- Limiti di grande segnale\\
- Amp. Op. reale\\
-- Amp. non invertente\\
- Stabilità di un circuito retroazionato\\
-- Margine di fase\\
- Oscillatori\\
-- Condizioni di Barkhausen\\
-- Oscillatore a ponte di Wien\\
--- Analisi qualitativa\\
--- Analisi quantitativa\\
- Comparatore\\
-- Comparatore "semplice"\\
-- Trigger di Schmitt\\
- Multivibratore astabile

\end{document}
